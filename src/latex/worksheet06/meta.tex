\documentclass{exam}
\usepackage{../../scripts/meta}

%%% Change these %%%%%%%%%%%%%%%%%%%%%%%%%%%%%%%%%%%%%%%%%%%%%%%%%%%%%%%%%%%%%%
\discnumber{4}
\title{}
\date{Berlekamp-Welsh, Countability, Self Reference, Counting}

%%%%%%%%%%%%%%%%%%%%%%%%%%%%%%%%%%%%%%%%%%%%%%%%%%%%%%%%%%%%%%%%%%%%%%%%%%%%%%%
\begin{document}
\maketitle
\rule{\textwidth}{0.15em}
\fontsize{12}{15}\selectfont
\thispagestyle{empty}


%%% General Notes %%%%%%%%%%%%%%%%%%%%%%%%%%%%%%%%%%%%%%%%%%%%%%%%%%%%%%%
\section{General Comments}
\begin{questions}
\item Counting: 1.1
\begin{itemize}
\item Make sure to give a general intro on stars \& bars problems, so that students can gain intuition on how to convert problems they see into stars \& bars. 
\item Ensure that students understand (in)distinguishable balls/(in)distinguishable bins. Like stars \& bars, students should be able to understand what problems correspond to balls \& bins/stars \& bars, etc.
\item Only do extra problems if you have time
\item Feel free to just skim over theorems
\end{itemize}
\item Combinatorial Proofs: \#1, 2, 3
\item Discrete Probability: \#1 (very basic for later in the week people)
\item For the Prius problem, draw all the parking spaces available in order to illustrate all the possibilities. Additionally, explain why the three Prius's must be grouped together as 1 Prius when solving the problem.
\item Try to form discrete probability as an extension of counting. for example in the cards problem, everything is the same as in counting, except with denominator of (52c5)
\item Make sure they understand everything on the diagram (although it may be more helpful for conditional)
\item If you've covered counting a lot, don't worry if you can't spend too much time on this section
\begin{itemize}
\item Disjoint vs Independent. The easiest way to emphasize this difference is through the venn diagrams above (disjoint events are almost always DEPENDENT).
\item Uniform probability space and how to calculate probabilities using set sizes in that space.
\item Feel free to come up with your own probability space problems. If students are shaky on the concepts, I recommend the following problems (only the solutions are below, but you can guess what the questions were.. Walrand seems to llike these divide-up-your-outcome-space problems
\item For problems on worksheet, just need to know that there are exactly 4 Kings in a deck of cards
Also, for each add that you are drawing exactly 2 cards in the manner described
\end{itemize}
% \item Monty Hall: Walk through the thing on the board so they see why the probability changes
% \begin{itemize}
% \item Explain how conditional probability works before letting students try to do this themselves (should cover in discrete probability section above)
% \item Important that students understand why grouping the doors together is essential to this problem
% \end{itemize}
\end{questions}

\section{Questions}


\end{document}
