\fbox{\begin{minipage}{16.3cm}
\textbf{Gaussian (Normal) Distribution}: $N(\mu, \sigma^2)$
\begin{itemize}
	\item Mean: $\mu$
	\item Variance: $\sigma^2$
	\item $f(x | \mu, \sigma^2) = \frac{1}{\sigma\sqrt{2\pi}} e^{\frac{-(x-\mu)^2}{2\sigma^2}}$
    \item If $X \mathtt{\sim} N(\mu, \sigma)$ then $b + X \mathtt{\sim} N(b + \mu, \sigma^2)$
    \item If $X \mathtt{\sim} N(\mu, \sigma)$ then $c X \mathtt{\sim} N(c \mu, c^2 \sigma^2)$
	\item If $X \mathtt{\sim} N(\mu_X, \sigma^2_X)$ and $Y \mathtt{\sim} N(\mu_Y, \sigma^2_Y)$ and $X$ and $Y$ are independent then 
        $$X + Y \mathtt{\sim} N(\mu_X + \mu_Y, \sigma^2_X + \sigma^2_Y)$$
	\item The CLT (Central Limit Theorem) states that for a sequence of iid random variables: $X_1, X_2, ... , X_n$, each with mean $\mu$ and variance $\sigma^2, \newline$
	\[\frac{X_1 + X_2 + ... + X_n - n\mu}{\sigma \sqrt{n}} \] 
	approaches the standard normal distribution $Z \mathtt{\sim} N(0, 1)$ 
\end{itemize}
\end{minipage}}


