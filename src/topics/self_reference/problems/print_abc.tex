\question Say that we have a program $M$ that decides whether any input 
program halts as long as it prints out the string “ABC” as the first 
operation that it carries out. Can such a program exist? Prove your answer.

 
 

 
 
\begin{solution} [1 in]
No. Such a program $M$ can not exist. We proceed as follows: we show 
that if such a program existed, the halting problem would be computable.

Consider any program $P$. If we wanted to decide if $P$ halted, we 
could simply create a new program $P'$ where $P'$ first prints out “ABC”, 
then proceed to do exactly what $P$ would. However, if $M$ existed, we 
could determine whether any program $P$ halted by converting it to a $P'$ 
and feeding it into $M$. This would solve the halting problem- but this 
is a contradiction, since by diagonalization we can prove that the 
halting problem is not solvable.

Therefore, $M$ can not exist!  

Below we have the Pseudocode for our program P.

\begin{verbatim}
    define M'(Program p, input i):
        Construct program p' which will first print('ABC') 
            then run p(i)
        run M(p', i)
    
    
    define M(Program p, input i):
        if p's first command is print('ABC'):
            if p(i) halts:
                return True
            else if p(i) loops forever:
                return False
\end{verbatim}

\end{solution}