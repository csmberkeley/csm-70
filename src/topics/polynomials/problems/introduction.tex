{\tabulinesep=1mm
\begin{tabu}{|p{16cm} |}
\hline
There are two fundamental properties of polynomials:
\begin{enumerate}
    \item A non-zero polynomial of degree $d$ has at most $d$ real roots.
    \item $d + 1$ distinct points uniquely define a polynomial of degree at most $d$.
\end{enumerate}
We can represent polynomials in 2 ways:
\begin{enumerate}
    \item \textbf{Coefficient Representation:} This representation is probably what you've seen before. We could represent a polynomial of degree $d$ like this:
    \[p(x) = a_dx^d + a_{d-1}x^{d-1} + a_{d-2}x^{d-2} + ... +a_1x + a_0\]
    As you can see, we need $d+1$ coefficients ($a_0...a_d$).
    \item \textbf{Value Representation:} Using the second property of polynomials, if we have a collection of $d+1$ distinct points
    \[(x_0, y_0), (x_1, y_1), ... , (x_d, y_d)\]
    we actually would have a unique polynomial of degree at most $d$.
\end{enumerate}
Going from coefficient to value representation is easy: evaluate the polynomial at $d+1$ different points. The other way is slightly harder. There are 2 ways that we can do this:
\begin{enumerate}
    \item \textbf{System of equations} We know the generic equation for a polynomial of degree $d$ is $p(x) = a_dx^d + a_{d-1}x^{d-1} + a_{d-2}x^{d-2} + ... +a_1x + a_0$. So we "plug in" our $d+1$ points into this equation to obtain $d+1$ equations in $d+1$ unknowns (the coefficients). 
    \item \textbf{Lagrange Interpolation} Solving a system of equations, especially over a modulus space can be difficult, so there's a formula that we can use to directly obtain the coefficient representation. 
   \[p(x) = \sum_{i=0}^{d}y_i  \frac{\prod_{j\neq i}x-x_j}{\prod_{j\neq i}x_i -x_j}\] 
\end{enumerate} \\
You may be used to doing polynomials as functions from $\mathbb{R}\to\mathbb{R}$. In this class, we work over $\textit{Galois Field}$. This means, for some prime $p$, our functions are from $ \{0,1,...,p-1\} to \{0,1,...,p-1\}$. To do this, we take the result mod $p$ when we evaluate the function, and our only possible inputs are integers. Addition, subtraction, multiplication, and exponentiation all work the same over Galois field as they do over the real numbers, however, division is slightly different. For example, the expression $\frac{x-1}{3} \mod 7$ would be valid in the real number space, but we can't have fractions in Galois fields. Remember that division in modulus is the same as multiplying by the inverse. So in GF(7) $\frac{x-1}{3} \equiv 5(x-1) \mod 7$. 
%$p(x) = \sum_{i=1}^{n+1} y_i \Delta_i(x)$, where $\Delta_i(x) = \frac{\prod_{j \neq i} x-x_j}{\prod_{j \neq i} x_i-x_j}$
\\
\hline
\end{tabu}
}
