{\tabulinesep=1mm
\begin{tabu}{|p{16cm} |}
\hline
\textbf{Geometric Distribution}: Geom($p$) \\

With the geometric distribution, we count the number of failures until the first success. The probability that the first success occurs on trial $k$ is:
\[ \P[X = k] = (1 - p)^{k - 1}*p, k > 0\]

\textit{Expectation:} \\
We can derive the geometric distribution from the binomial distribution. $\E(X)$ is the number of trials until the first success occurs, including that first success. There are two cases:
\begin{enumerate}
	\item{The first success occurs, with probability $p$.}
	\item{We obtain a failure, with probability $1-p$, meaning that we are back where we started but already used one trial.}
\end{enumerate}
Putting this together, we get:
$$E(X) = p * 1 + (1-p) * (1 + E(X)) \implies E(X) = \frac{1}{p}$$

\textit{Variance:} \\
$$var(X) = \frac{1-p}{p^2}$$
We note that the geometric distribution is memoryless; its probability distribution is independent of its history. 
\[\P(X > t+s | x > t) = \P(X > s) for all r, t \geq 0 \]
\\
\hline
\end{tabu}
}