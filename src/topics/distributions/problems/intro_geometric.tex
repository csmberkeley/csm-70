\vspace{2 mm}
\textbf{Geometric Distribution}: Geom($p$)

With the geometric distribution, we count the number of failures until the first success. For example, we could count the number of rolls of a dice until we roll a 6. The probability that the first success occurs on trial $k$ is:
\[ \P[X = k] = (1 - p)^{k - 1}*p, k > 0\]

In what way can we derive the geometric distribution from the binomial distribution?

\textit{Expectation:} \\
We know that $E(X)$ is the number of trials until the first success occurs, including that first success. There are two cases:
\begin{enumerate}
	\item{The first success occurs, with probability $p$}
	\item{We obtain a failure, with probability $1-p$, meaning that we are back where we started but already used one trial}
\end{enumerate}

Putting this together, we get:

$$E(X) = p * 1 + (1-p) * (1 + E(X)) \implies E(X) = \frac{1}{p}$$

\textit{Variance:} \\
$$var(X) = \frac{1-p}{p^2}$$
