\vspace{2 mm}
\textbf{Poisson Distribution:} Pois($\lambda$)
The Poisson distribution is an approximation of the binomial distribution under two conditions:
\begin{itemize}
	\item{$n$ is very large}
	\item{$p$ is very small}
\end{itemize}

Let $\lambda = np$ represent the "rate" at which some event occurs. We usually use this distribution when these events are rare, such as a lightbulb failing. 

The probability of $k$ occurrences is 
$$ \P[X = k] = \frac{e^{-\lambda} * \lambda^k}{k!} $$

It turns out that the expectation and variance of the Poisson distribution are both equal to $\lambda$. This will be clear after we walk through the derivation of the Poisson distribution.

\textit{Derivation}: \\
Recall, $\lambda = np$. Also, recall from calculus we have $\lim_{n \rightarrow \infty} \left(1 + \frac{x}{n}\right)^n = e^x$, implying that $\lim_{n \rightarrow \infty} \left(1 -\frac{\alpha}{n}\right)^n = e^{- \alpha}$. We will also use the fact that for large $n$, $\frac{n!}{(n-k)!} \approx n^k$. We will use these facts below.
\begin{align}
\P[X = k] &= {n \choose k} * p^k * (1 - p)^{n - k} \\ 
&= \frac{n!}{k! * (n - k)!} * p^k * (1 - p)^{n - k} \\
&\approx \frac{n^k * p^k}{k!} * (1 - \frac{\lambda}{n})^{n-k} \\
&\approx \frac{\lambda^k * e^{-\lambda}}{k!}
\end{align}

Since we started with a binomial distribution, our expectation and variance should remain the same.

\textit{Expectation}: \\
Since the expectation of a binomial is $np$, and we set $\lambda = np$, our expectation is also $E(X) = np$. We can also show this from scratch:

\begin{align*}
\E(X) &= \sum_{k = 0}^{\infty} k* \frac{e^{-\lambda} * \lambda^k}{k!} \\ \nonumber
&= \sum_{k = 1}^{\infty}k * \frac{e^{-\lambda} * \lambda^k}{k!} \\
&= e^{-\lambda} * \lambda * \sum_{k = 1}^{\infty} \frac{\lambda^{k - 1}}{(k - 1)!} \\
&=  e^{-\lambda} * \lambda * \sum_{k = 0}^{\infty} \frac{\lambda^{k}}{k!} \\
&= e^{-\lambda} * \lambda * e^{\lambda} \\
&= \lambda 
\end{align*}

\textit{Variance}: \\
For variance, it is much easier to start with the binomial case and reason from there. The variance of a binomial is $np(1-p)$, which looks like $\lambda (1-p)$. However, we started with the assumption that $p$ is very small, so we can assume $(1-p)$ is very close to $1$ and thus $\lambda (1-p)$ is very close to $\lambda$. Therefore, $var(X) = \lambda$. 