\question We have a die with 6 faces of values $1, 2, 3, 4, 5, 6$.

\begin{enumerate}[label=(\alph*)]
\item Develop a 99\% confidence interval for the value of n samples. 
\begin{solution}[3cm]
Consider the $99\%$ confidence interval for the average of $n$ rolls of a standard die, $X = \frac{1}{n} \sum_{i=1}^n D_i$. 
We can model it using the normal distribution with a mean $\E[X] = \frac{n\E[D_i]}{n} = \E[D]= 3.5$ and a variance $\text{Var}[X] = (\frac{1}{n})^2\text{Var}[X] = (\frac{1}{n})^2 n \text{Var}[D_i] = \frac{\text{Var}[D_i]}{n} = \frac{35}{12n}$.
We can now find the z-score (normalized $x$ value) of the $99\%$ confidence interval is $\Phi^{-1} (0.995) = 2.576$. 
Recall that the relationship between $X$ and $Z$ (the standard normal random variable) is $Z = \frac{X - \E[X]}{\sqrt{Var[(X)}}$. Thus, $x = \E[X] \pm \Phi^{-1}(0.995)\sqrt{\text{Var}(X)} = 3.5 \pm 2.576\sqrt{\frac{35}{12n}}$
\end{solution}

\item Now, we say the die's face values are consecutive integers, but we do no��t know the starting number. The values are shifted over by some $k$; for example, if $k = 6$, the die faces would take on the values 
$7, 8, 9, 10, 11, 12$. If we observe that the average of the $n$ samples is 15.5, develop a 99\% confidence 
interval for the value of $k$. 

\begin{solution}
We can now say that the confidence interval for $k$ is given by $15.5$ $-$ confidence interval for $X$.
\end{solution}
\end{enumerate}
