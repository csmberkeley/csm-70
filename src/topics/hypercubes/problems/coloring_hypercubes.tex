\question \textbf{Coloring Hypercubes} \newline Let $G = (V, E)$ be 
an undirected graph. $G$ is said to be $k$-vertex-colorable if it is 
possible to assign one of $k$ colors to each vertex of $G$ so that no 
two adjacent vertices receive the same color. $G$ is $k$-edge-colorable 
if it is possible to assign one of $k$ colors to each edge of $G$ so 
that no two edges incident on the same vertex receive the same color. \newline
Show that the $n$-dimensional hypercube is 2-vertex-colorable for 
every $n$.
\begin{solution}[1in] 

\textbf{Direct proof:}
Notice that 2-vertex colorable is another way to say bipartite. Thus we are looking for a way to split the vertices into two sets $U$ and $V$ such that every edge in the graph connects some vertex in $U$ to some vertex in $V$. Let $U$ be the set of vertices with odd parity and let $V$ be the set of vertices with even parity. An edge will always connect a vertices of different parities since an edge corresponds with a bit flip in the bit-definition. Thus every edge crosses $U$ and $V$.\newline

\textbf{Inductive proof:}
Base case: For $n = 1$, the hypercube is a single edge. If we color 
one vertex red and the other blue we have a 2-vertex coloring, since 
the adjacent vertices are colored differently. 

Inductive Step: Assume we’ve shown this to hold for $n$-dimensional 
hypercubes, we will show this holds for $n + 1$-dimensional hypercubes. 
Recall that we can define an $n + 1$ dimensional hypercube as two 
$n$-dimensional hypercubes where every vertex $i$ in the first 
hypercube is connected to vertex $i$ in the second hypercube. 
Considering this definition, for a given $n + 1$ dimensional 
hypercube, let $H_0$, $H_1$ denote the first and second n-dimensional 
hypercubes, respectively. By the inductive hypothesis, we assume that 
$H_0$ is 2-vertex colorable, and therefore there exists some coloring 
scheme which is a legal 2-vertex coloring of $H_0$. Given this coloring, 
we will color $H_1$ in the opposite coloring scheme which, given a color 
of vertex $ i$ in $H_0$ assigns the opposite color to the vertex $i$ in 
$H_1$. Since we colored the vertices in both $H_0$ and $H_1$, we have 
colored all the vertices in the hypercube. 

It remains to show that this coloring scheme is legal. Assume, for purpose of contradiction that there is a given 
vertex $i$ in $H_1$ which has an adjacent neighbor colored with the 
same color. If that neighbor is in $H_1$, then this means that the 
coloring of $H_1$ is not legal. Observe that if a coloring scheme is 
a legal 2-vertex coloring on some graph $G$, then the opposite 
coloring scheme is also a legal 2-vertex coloring on $G$. Since we 
colored $H_1$ with the opposite scheme of $H_0$, and $H_0$ is identical 
to $H_1$, this implies that the coloring of $H_1$ is a legal 2-vertex 
coloring, which contradicts having two adjacent vertices in $H_1$ 
sharing the same color. If the neighbor is in $H_0$, then we know, 
by definition of the $n + 1$-dimensional hypercube, that the neighbor 
must be $i$ in $H_0$. In our coloring however, we colored $i$ in $H_0$ 
and $i$ in $H_1$ in opposite colors, which again contradicts our 
assumption. Similarly, we can show for the case where $i$ is in $H_0$
\end{solution}
