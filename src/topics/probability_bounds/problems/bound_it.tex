\question A random variable $X$ is always strictly larger than $-100$. You know 
that $\E(X) = -60$. Give the best upper bound you can on $\P[X \geq -20]$.
    %\vspace{2.5cm}
\begin{solution}[2cm]
We do not have the variance of $X$, so Chebyshev's bound is not applicable here. 
This leaves us with just Markov's Inequality. But, the Markov bound only applies on a non-negative random variable, so we must shift $X$ somehow. Define a random variable $Y = X + 100$; $Y$ is strictly larger than 0. Then, $\E(Y) = \E(X +100) = \E(X)+100 = -60+100 = 40$. The upper bound on $X$ that we want can be calculated via $Y$:
					
$\P[X \geq -20] = \P[Y \geq 80] \leq \frac{\E(Y)}{80} = \frac{1}{2}$
					
Hence, the best upper bound on $\P[X \geq -20]$ is $\frac{1}{2}$. 
\end{solution}