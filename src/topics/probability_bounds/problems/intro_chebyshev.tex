{\tabulinesep=1mm
\begin{tabu}{|p{16cm} |}
\hline
\vspace{2 mm}
\textbf{Chebyshev's Inequality } \newline
For a random variable $X$ with expectation  $\E(X) = \mu$, and any  
$\alpha > 0$:
\[\P[|X - \mu | \geq \alpha] \leq \frac{\var(X)}{\alpha^2}\]
Chebyshev's Inequality can be used to estimate the mean of an unknown distribution. \newline
Often times we do not know the true mean, so we take many samples $X_1, X_2, ... , X_n$. \newline 
Our sample mean is thus $S_n = \frac{X_1 + X_2 + ... + X_n}{n}$ \\
We can upper-bound the probability of that our sample mean deviates too much from our true mean as: 
\[P(|\hat{\mu} - \mu| > \epsilon) \leq \delta\]
where $\epsilon$ is known as our error and $\delta$ is known as our confidence. \newline
\\
\hline
\end{tabu}}
