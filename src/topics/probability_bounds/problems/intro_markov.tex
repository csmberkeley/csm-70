{\tabulinesep=1mm
\begin{tabu}{|p{16cm} |}
\hline
\vspace{2 mm}
\textbf{Markov's Inequality } \newline
For a non-negative random variable $X$ with expectation $\E(X) = \mu$, 
and any  $\alpha > 0$:
\[\P[X \geq \alpha] \leq \frac{\E(X)}{\alpha}\]
\vspace{2 mm}
\textbf{Proof of Markov's Inequality}\newline

\[E(X) = \sum_a a * Pr[X = a] \]
\[\geq \sum_{a \geq \alpha} a *  Pr[X = a] \]
\[\geq \alpha \sum_{a \geq \alpha} Pr[X = a] \]
\[= \alpha Pr[X \geq a]\]

\textbf{Alternate Proof of Markov's Inequality} \newline
Consider the indicator random variable $Y$ which equals $1$ if $X \geq a$ and $0$ otherwise. \newline
Now consider the relationship between $X$ and $aY$ \newline
\begin{itemize}
	\item If $X < a$, then $Y = 0$, which means $aY = a * 0 = 0$ \newline
	Because $X$ is a non-negative random variable, $X \geq 0$, so $aY \leq X$ in this case. 
	\item If $X \geq a$, then $Y = 1$, which means $aY = a * 1 = a \leq X$ \newline
\end{itemize}
Thus, we have $aY \leq X$. \newline
We take expectation on both sides to get: 
\[E[aY] \leq X\]
\[aE[Y] \leq E[X]\]
\[E[Y] \leq \frac{E[X]}{a} \]
Now we note that the expectation of an indicator random variable is the probability that it is equal to $1$ and we have the proof:
\[P(X \geq a) \leq \frac{E[X]}{a} \]
\vspace{2 mm}
\\
\hline
\end{tabu}
}

