\question 
\begin{enumerate}[label=(\alph*.)]
\item
What does it mean for a set to be countably infinite? 
\begin{solution}
There is a bijection between the set and the natural numbers (or any other countable set).
\end{solution}

\item
Do $\mathbb{N}$ and $\mathbb{Z}^+$ have the same cardinality? Does adding one element change cardinality?
\begin{solution}
Do the hotel argument. Just take each person starting at some arbitrary point $k$, and slide them over by one. In other words, map everybody at some point $p \ge k$ to $p + 1$, and then slide the newcomer into spot $k$.

No, adding in one more point did not change the cardinality of the set.
\end{solution}
\item
Cantor-Bernstein Theorem: Suppose there is an injective function from set A to set B and there is an injective function from set B to set A. Then there is a bijection between A and B.
Use this theorem to prove that Q is countable.
\begin{solution}
This is used when proving that rational numbers are countable. 
We know that $|\mathbb{N}| \le |Q|$ because every natural number is a rational number.
Just need to show that $|Q| \le |\mathbb{N}|$. Do the spiral proof (which is presented in the notes).
\end{solution}

\end{enumerate}