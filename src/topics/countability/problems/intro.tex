{\tabulinesep=1mm
\begin{tabu}{|p{16cm} |}
\hline
\begin{enumerate}[label=(\alph*.)] 
\item 
Cardinality is the number of elements in a set. We define a set as countably infinite if it has the same cardinality as the natural numbers (or any countable set). 

%For instance, $\mathbb{Z}^+$ is countably infinite because $\mathbb{N}$ and $\mathbb{Z}^+$ have the same cardinality. Do the hotel argument. Just take each person starting at some arbitrary point $k$, and slide them over by one. In other words, map everybody at some point $p \ge k$ to $p + 1$, and then slide the newcomer into spot $k$.

%No, adding in one more point did not change the cardinality of the set.
\item 
We can prove a set is countable by finding a bijection between it and any countable set. A few classic examples are the hotel argument to show that $\mathbb{Z}^+$ is countable, and the spiral argument to show that $\mathbb{Q}$ is countable, both included in your notes. 
\item 
To prove a set is uncountable, we can either find a bijection between it and an uncountable set or use the Cantor Diagonalization proof, included in your notes. 

%Cantor-Bernstein Theorem: Suppose there is an injective function from set A to set B and there is an injective function from set B to set A. Then there is a bijection between A and B.
%Use this theorem to prove that Q is countable.
%\begin{solution}
%This is used when proving that rational numbers are countable. 
%We know that $|\mathbb{N}| \le |Q|$ because every natural number is a rational number.
%Just need to show that $|Q| \le |\mathbb{N}|$. Do the spiral proof (which is presented in the notes).
%\end{solution}

\end{enumerate}
\\
\hline
\end{tabu}
}