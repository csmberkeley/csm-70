\question Mr. and Mrs. Brown decide to continue having children until they either have their first girl or until
they have three children. Assume that each child is equally likely to be a boy or a girl, independent
of all other children, and that there are no multiple births. Let G denote the numbers of girls that
the Browns have. Let C be the total number of children they have.

\begin{enumerate}[label=(\alph*)]
\item Determine the sample space, along with the probability of each sample point.
\begin{solution}[.75 cm]
The sample space is the set of all possible sequences of children that the Browns can have: $\omega = \{g, bg, bbg, bbbg\}$. The probabilities of these sample points are:
\[\P(g) = \frac{1}{2}\]
\[\P(bg) = \frac{1}{2} + \frac{1}{2} = \frac{1}{4}\]
\[\P(bbg) = {\frac{1}{2}}^3= \frac{1}{8}\]
\[\P(bbb) = {\frac{1}{2}}^3= \frac{1}{8}\]
\end{solution}

\item Compute the joint distribution of G and C. Fill in the table below.
\begin{center}
\begin{tabular}{|l|l|l|l|}
\hline
\textbf{}      & \textbf{C = 1} & \textbf{C = 2} & \textbf{C = 3} \\ \hline
\textbf{G = 0} &                &                &                \\ \hline
\textbf{G = 1} &                &                &                \\ \hline
\end{tabular}
\end{center}
\begin{solution}[.75 cm]
\begin{center}
\begin{tabular}{|l|l|l|l|}
\hline
\textbf{}      & \textbf{C = 1}             & \textbf{C = 2} & \textbf{C = 3} \\ \hline
\textbf{G = 0} & 0                          & 0              & \P(bbb) = 1/8   \\ \hline
\textbf{G = 1} & \P(g) = 1/2 & \P(bg) = 1/4    & \P(g) = 1/8     \\ \hline
\end{tabular}
\end{center}
\end{solution}

\item Use the joint distribution to compute the marginal distributions of G and C and confirm that the values are as you'd expect. Fill in the tables below. 
\begin{center}
\begin{tabular}{|l|l|}
\hline
\textbf{\P(G = 0)} &  \\ \hline
\textbf{\P(G = 1)} &  \\ \hline
\end{tabular}
\begin{tabular}{|l|l|l|}
\hline
\textbf{\P(C = 1)} & \textbf{\P(C = 2)} & \textbf{\P(C = 3)} \\ \hline
                  &                   &                   \\ \hline
\end{tabular}
\end{center}
\begin{solution}[.75 cm]
Marginal distribution for G:
\[\P(G = 0) = 0 + 0 + \frac{1}{8} = \frac{1}{8}\]
\[\P(G = 1) = \frac{1}{2} + \frac{1}{4} + \frac{1}{8} = \frac{7}{8}\]
Marginal distribution for C:
\[\P(C = 1) = 0 + \frac{1}{2} = \frac{1}{2}\]
\[\P(C = 2) = 0 + \frac{1}{4} = \frac{1}{4}\]
\[\P(C = 3) = \frac{1}{8} + \frac{1}{8} = \frac{1}{4}\]
\end{solution}

\item Are G and C independent?
\begin{solution}[3 mm]
No, G and C are not independent. If two random variables are independent, then
\[\P(X = x, Y = y) = \P(X = x)\P(Y = y)\]
To show this dependence, consider an entry in the joint distribution table, such as
$\P(G = 0, C = 3) = \frac{1}{8}$. This is not equal to $\P(G = 0)\P(C = 3) = \frac{1}{8}*\frac{1}{4} = \frac{1}{32}$, so the random variables are not independent.
\end{solution}

\item What is the expected number of girls the Browns will have? What is the expected number of
children that the Browns will have?
\begin{solution}[1 cm]
We can apply the definition of expectation directly for this problem, since we?ve computed the
marginal distribution for both random variables.
\[\E(G) = 0*\P(G = 0) + 1*\P(G = 1) = \frac{7}{8}\]
\[\E(C) = 1*\P(C = 1) + 2*\P(C = 2) + 3*\P(C = 3) = 1*\frac{1}{2} + 2*\frac{1}{4} + 3*\frac{1}{4} = \frac{7}{4}\]
\end{solution}
\end{enumerate}
