\noindent\fbox{\begin{minipage}{\dimexpr\textwidth-2\fboxsep-2\fboxrule\relax}
\vspace{5 mm}
 \textbf{Chinese Remainder Theorem}: The Chinese Remainder theorem says that a
 sequence of remainders with pairwise coprime divisors defines a unique remainder modulo the product of those divisors. Formally,
 if $x$ can be expressed as
 \begin{align*}
     x &\equiv a_1 (\mod m_1) \\
     x &\equiv a_2 (\mod m_2)
 \end{align*} 
 where $m_1$ and $m_2$ are relatively prime to each other, CRT tells us that there is an unique number mod $m_1m_2$ that satisfies this equation.  \\ 
 In simple cases, we can often use extended Euclid's algorithm in simple cases to find $x$. However, a failsafe equation is given by: \\
 $x = \sum_{i=1}^k a_i b_i \mod N$, where $b_i$ are defined as $\left(\frac{N}{n_i} \right) \left(\frac{N}{n_i} \right)^{-1}_{\mod n_i}$ and $N = n_1 \cdot n_2 ... \cdot n_k$.
\end{minipage}}
