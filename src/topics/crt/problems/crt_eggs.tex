\question The supermarket has a lot of eggs, but the manager is not sure exactly how many he has. When he splits the eggs into groups of 5, there are exactly 3 left. When he splits the eggs into groups of 11, there are 6 left. What is the minimum number of eggs at the supermarket?
\begin{solution}[1.5 in]
We have that $x \equiv 3 \mod 5$ and $x \equiv 6 \mod 11$. We can use the Chinese Remainder Theorem to solve for x.

Recall from the note on modular arithmetic, the solution to $x$ is defined as $x = \sum_{i=1}^k a_i b_i \mod N$, where $b_i$ are defined as $\left(\frac{N}{n_i} \right) \left(\frac{N}{n_i} \right)^{-1}_{\mod n_i}$ and $N = n_1 \cdot n_2 ... \cdot n_k$.

In our case, $a_1 = 3, a_2 = 6, n_1 = 5$ and $n_2 = 11$. 

$b_1 = \left(\frac{55}{5} \right) \left(\frac{55}{5} \right)^{-1}_{\mod 5} = 11 \cdot 11^{-1}_{\mod 5} = 11 * 1 = 11$

$b_2 = \left(\frac{55}{11} \right) \left(\frac{55}{11} \right)^{-1}_{\mod 11} = 5 \cdot 5^{-1}_{\mod 11} = 5 * 9 = 45$

Therefore, $x \equiv 3 \cdot 11 + 6 \cdot 45 (\mod 55) = 28$

You can quickly verify that $28$ indeed satisfies both conditions.

\end{solution}
