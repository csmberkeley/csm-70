\question For what values of the parameters are the following functions 
probability density functions? What is the expectation and variance of 
the random variable that the function represents?
\begin{enumerate}[label=(\alph*)]
    \item $f(x) = \begin{cases} ax & 0 < x < 1 \\ 0 & \text{otherwise} 
    \end{cases}$
 	\begin{solution}[5.2cm]
For a function to represent a probability density function, we need to 
have that the integral of the function from negative infinity to positive 
infinity to equal 1 and for f(x) to be greater than or equal to 0. So 
we need integral over $(-\infty, \infty)$ $ \int_{-\infty}^\infty f(x) = 1 
= \int_{0}^{1} ax dx = \frac{ax^2}{2} \mid_{0}^{1} = 1 \iff  \frac{a}{2} - 
0 = 1 \iff a = 2$ \\
For RV $Y$ with pdf = $f(x)$, 
$\E(Y) = \int_{-\infty}^{\infty} x \times f(x) dx = \int_{0}^{1} x \times 
2x dx = \frac{2x^3}{3} \mid_{0}^{1} = \frac{2}{3} - 0 = \frac{2}{3}$ \\
$\var(Y) = \int_{-\infty}^{\infty} x^2 f(x) dx - \E[Y]^2 = 
\int_{0}^{1} x^2 2x dx - \frac{4}{9}= \int_{0}^{1} 2x^3 dx - 
\frac{4}{9}= \frac{x^4}{2} \mid_{0}^{1} = \frac{1}{2} - 0 -\frac{4}{9} 
= \frac{1}{18}$
	\end{solution}
    
\item $f(x) = \begin{cases} -2x & \text{if } a < x < b\ (a = 0 \lor b 
= 0)\\ 0 & \text{otherwise} \end{cases}$
\begin{solution}[5.2cm]
Again we need $f(x) \ge 0$, so here $a, b \le 0$, so $b=0$.
        Then 
$\int_{a}^{0} f(x) dx = 1 = \int_{a}^{0} -2x dx = \frac{-2x^2}{2} \mid_{a}^{0}  \\
= 0 - (\frac{-2a^2}{2}) = \frac{2a^2}{2} = 1 \iff a^2 = 1 \iff a \\
= \pm 1 \implies a = -1$. \\
\end{solution}

% For RV $Y$ with pdf = $f(x)$, \\
% $\E(Y) = \int_{-\infty}^{\infty} x \times f(x) dx = \int_{-1}^{0} x \times 
% (-2x) dx =  \frac{-2x^3}{3} \mid_{-1}{0} = 0 - (\frac{(-2)(-1)^3}{3}) = 
% -\frac{2}{3}$. \\
% $Var(Y) = \int_\infty^{-\infty} x^2*f(x) dx = \int_0^{-1} x^2*(-2x) dx
% = -x^4/2 |_0^{-1} = 0 - (-(-1)4)/2 = \frac{1}{2}$ \\
\question
$f(x) = \begin{cases} c & -30 < x < -20 \lor -5 < x < 5 \lor 60 < x < 
70 \\ 0 & \text{otherwise} \end{cases}$ \\
Don’t worry too much about calculations, but you should be able to set 
up the equations
\begin{solution}[5.0cm]
We need $\int_{-\infty}^{\infty} f(x) dx = 1$ and $f(x) \ge 0$. So $c \ge 0$ \\
$\int_{\infty}^{\infty} f(x) dx = 1 = \int_{-30}^{-20} c dx + \int_{-5}^{5} c dx
+ \int_{60}^{70} c dx = cx\mid_{-30}^{-20} + cx\mid_{-5}^{5} + cx\mid_{60}^{70} \\
= 10c + 10c + 10c = 30c = 1 \implies \frac{1}{30}$

\begin{align*}
    E(Y) &= \int_{-\infty}^{\infty} x * f(x) dx \\
         &= \int_{-30}^{-20} xc dx + \int_{-5}^{5} xc dx + \int_{60}^{70} xc dx \\
         &= \frac{x^2c}{2}\mid_{-30}^{-20} + \frac{x^2c}{2}\mid_{-5}^{5} + \frac{x^2c}{2}\mid_{60}^{70} \\
         &= \frac{(-30)^2c}{2} - \frac{(-20)^2c}{2} + \frac{5^2c}{2} - \frac{(-5)^2c}{2} + \frac{70^2c}{2} - \frac{60^2c}{2} \\
         &= 900c = \frac{900}{30} = 30 \\\\
    E(Y^2) &= \int_{-\infty}^{\infty} x^2f(x) dx \\
           &= \int_{-20}^{30} x^2c dx + \int_{-5}^{5} x^2c dx + \int_{60}^{70} x^2c dx \\
           &= \frac{x^3c}{3} \mid_{-30}^{20} + \frac{x^3c}{3}\mid_{-5}^{5} + \frac{x^3c}{3}\mid_{60}^{70} \\
           &= \frac{(-30)^3c}{3} - \frac{(-20)^{3}c}{3} + \frac{5^3c}{3} - \frac{(-5)^3c}{3} + \frac{70^3c}{3} - \frac{60^3c}{3} \\
           &= \frac{108250c}{3} = 1202.77\ldots
\end{align*}
\end{solution}
\end{enumerate}
