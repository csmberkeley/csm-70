\question \textbf{Stanford Cinema} \\*
You have a database of an infinite number of movies. Each movie has a 
rating that is uniformly distributed in {0, 1, 2, 3, 4, 5} independent of all other movies. You want to find two 
movies such that the sum of their ratings is greater than 7.5 (7.5 is 
not included).
\begin{enumerate}[label=\alph*)]
\item
A Stanford student chooses two movies each time and calculates the sum 
of their ratings. If is less than or equal to 7.5, the student throws 
away these two movies and chooses two other movies. The student stops 
when he/she finds two movies such that the sum of their ratings is 
greater than 7.5. What is the expected number of movies that this 
student needs to choose from the database?

\begin{solution}[4cm]
Each time when the Stanford student chooses two movies, there are $6^2 
= 36$ different possible pairs of ratings. We know that there are 4 pairs 
whose sum is greater than $7.5$. Therefore, 
the probability that in a single trial, the Stanford student gets two 
movies such that the sum of their ratings is greater than $7.5$ is 
$\frac{1}{9}$. Then the number of times that the student needs to 
pick movies is geometrically distributed with mean $9$. 
Then the expected number of movies that the student needs to choose 
is $18$.
\end{solution}

\item
A Berkeley student chooses movies from the database one by one and 
keeps the movie with the highest rating. The student stops when he/she 
finds the sum of the ratings of the last movie that he/she has chosen 
and the movie with the highest rating among all the previous movies is 
greater than 7.5. What is the expected number of movies that the 
student will have to choose?

\begin{solution}[9cm]
To solve this problem, let's create a Markov chain with 5 states. One state, $S_0$ will represent starting or having a movie with rating $0, 1$ or  $2$ in hand. This state is special because no matter what the next state is, we won't be able to draw a movie on the next round that will have the sum of the ratings be more than $8$. We will have a "Done" state, $S_4$ that represents finding 2 movies that have rating more than $8$. The other 3 states $S_1, S_2, S_3$ will represent having movies with rating $3, 4, 5$ in hand, respectively. From each of these states, note that we have some probability of drawing some movie on the next step that puts the total greater than 8. This is the transition matrix representing the chain:

\[
\begin{bmatrix}
1/2 & 1/6 & 1/6 & 1/6 & 0\\
0 & 2/3 & 1/6 & 0 & 1/6\\
0 & 0 & 2/3 & 0 & 1/3\\
0 & 0 & 0 & 1/2 & 1/2\\
0 & 0 & 0 & 0 & 1\\
\end{bmatrix}
\]

The corresponding hitting time equations are:
\begin{gather*}
\beta(4) = 0\\
\beta(3) = \frac{1}{2}\beta(3) + \frac{1}{2}\beta(4)\\
\beta(2) = \frac{2}{3}\beta(3) + \frac{1}{3}\beta(4)\\
\beta(1) = \frac{2}{3}\beta(1) + \frac{1}{6}\beta(3) + \frac{1}{6}\beta(3)\\
\beta(0) = \frac{1}{2}\beta(0) + \frac{1}{6}\beta(1) + \frac{1}{6}\beta(2) + 
\frac{1}{6}\beta(3)
\end{gather*}
Solving these, we find $\beta(0) = \frac{31}{6}$
This shows that the Berkeley student is smarter than the Stanford student.
\end{solution}

\end{enumerate}