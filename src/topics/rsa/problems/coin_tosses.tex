\question \textbf{Coin tosses over text messages} \newline
You and one of your friends want to get your hands on the new gadget 
that’s coming out. One of you has to wait in line overnight so that 
you have a chance to get the gadgets while they last. In order to 
decide who this person should be, you both agree to toss a coin. But 
you won’t meet each other until the day of the actual sale and you 
have to settle this coin toss over text messages (using your old gadgets). 
Obviously neither of you trusts the other person to simply do the coin 
toss and report the results.\newline
					
How can you use RSA to help fix the problem?	
\begin{solution}
Firstly, we need some way to create two events of equal probability. One way to do this is have both of you flip a coin, and if you both get the same coin (i.e. "HH" or "TT") then one of you has to wait in line, and if you both get different coins (i.e. "HT" or "TH") then the other one of you needs to wait in line. Since each combination of two flips has probability $\frac{1}{4}$, each of you has a $2 \cdot \frac{1}{4} = \frac{1}{2}$ chance of having to wait in line.

Some form of encryption is necessary in this question, because otherwise, you could send your non-encrypted result to your friend and they could simply lie and say that they got the result that would cause you to have to go wait in line. Consider the following abstraction:
\begin{enumerate}
	\item I flip a coin, write down my result on a piece of paper, and turn over the piece of paper
	\item I give you the paper, you flip your coin, and then turn over my paper
\end{enumerate}
The main idea of the above scheme boiled down to the fact that I was able to give you my result without you being able to see it. After flipping my coin and writing down the result, I'm unable to change my flip, therefore I was forced to "commit" to my answer. Since you weren't able to see my coin, you had no choice but to actually flip the coin, as you had no additional information.

Let's implement this with RSA.
\begin{enumerate}
	\item You generate a public key $(N, e)$, and flip your coin. If you got heads, encrypt some random word beginning with the letter "H". If you got tails, encrpyt some random word beginning with the letter "T". The reasoning for this will be evident in the next step.
	\item Send your encrypted message to your friend, along with your public key. Sending the public key is important, since it forces you to commit to the result that you got (if you were to send your public key after your friend flipped their coin, you could calculate a public key and private key such that your encrypted message decrypts to whatever you want it to).
	\item Since your friend currently has no information about your flip, as you can't decrypt with just the public key, they have no choice but to just flip their coin and relay their result back to you.
	\item Now, you send your friend your private key. With your public key and private key, they can decrypt your message to see what your original word was, telling them whether you got heads or tails.
\end{enumerate}
\end{solution}

\clearpage