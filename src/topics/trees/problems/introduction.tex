\noindent\fbox{\begin{minipage}{\dimexpr\textwidth-2\fboxsep-2\fboxrule\relax}
\vspace{3 mm}
If complete graphs are “maximally connected,” then trees are the opposite: 
Removing just a single edge disconnects the graph! Formally, there are a 
number of equivalent definitions for identifying a graph $G = (V,E)$ as 
a tree. 
\vspace{1 mm}
\end{minipage}}

\vspace{1 mm}

Assume $G$ is connected. There are 3 other properties we can use to 
define it as a tree.

\vspace{2 mm}

\begin{enumerate}[itemsep=5mm]
\item $G$ contains \line(1,0){50} cycles.
\item $G$ has \line(1,0){50}  edges.
\item Removing any additional edge will \line(1,0){90} 
\end{enumerate}

\begin{solution}
no, $n-1$, disconnect $G$
\end{solution}

\vspace{4 mm}

One additional definition: \newline

\begin{enumerate}[itemsep=5mm]
\setcounter{enumi}{3}
\item $G$ is a tree if it has no cycles and \line(1,0){150}
\end{enumerate}

\begin{solution}
adding any edge creates a cycle
\end{solution}

\vspace{1mm}

\noindent\fbox{\begin{minipage}{\dimexpr\textwidth-2\fboxsep-2\fboxrule\relax}
\vspace{3 mm}
\textbf{Theorem}: $G$ is connected and contains no cycles if and only 
if $G$ is connected and has $n - 1$ edges.
\vspace{1 mm}
\end{minipage}}