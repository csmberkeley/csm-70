\question Now show that if a graph satisfies either of these two 
properties then it must be a tree:
\begin{enumerate}[label=\alph*]
\item If for every pair of vertices in a graph they are connected by 
exactly one simple path, then the graph must be a tree.
\begin{solution}[1 in]
Assume we have a graph with the property that there is a unique simple 
path between every pair of vertices. We will show that the graph is a 
tree, namely, it is connected and acyclic. First, the graph is connected 
because every pair of vertices is connected by a path. Moreover, the 
graph is acyclic because there is a unique path between every pair of 
vertices. More explicitly, if the graph has a cycle, then for any two 
vertices $x$, $y$ in the cycle there are at least two simple paths 
between them (obtained by going from $x$ to $y$ through the right or 
left half of the cycle), contradicting the uniqueness of the path. 
Therefore, we conclude the graph is a tree.
\end{solution}

\item If the graph has no simple cycles but has the property that the 
addition of any single edge (not already in the graph) will create a 
simple cycle, then the graph is a tree.
\begin{solution}[1 in]
Assume we have a graph with no simple cycles, but adding any edge will 
create a simple cycle. We will show that the graph is a tree. We know 
the graph is acyclic because it has no simple cycles. To show the graph 
is connected, we prove that any pair of vertices $x$, $y$ are connected 
by a path. We consider two cases: If $(x, y)$ is an edge, then clearly 
there is a path from $x$ to $y$. Otherwise, if $(x,y)$ is not an edge, 
then by assumption, adding the edge $(x,y)$ will create a simple cycle. 
This means there is a simple path from $x$ to $y$ obtained by removing 
the edge $(x, y)$ from this cycle. Therefore, we conclude the graph is 
a tree.
\end{solution}
\end{enumerate}

