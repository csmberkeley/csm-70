\question You have a deck of 52 cards. What is the probability of:
\begin{enumerate}[label=(\alph*)]

\item Drawing 2 Kings with replacement?
\begin{solution}
There are $52*52$ combinations of two 
cards. There are $4*4$ that are 2 Kings, so the probability is 
$\frac{1}{13^2}$.
 \end{solution}
 
\item Drawing 2 Kings without replacement?
\begin{solution}
There are ${52 \choose 2} = 52*51$ pairs of cards possible without replacement. 
Of these $4*3$ represent pairs of kings. Therefore the 
probability is $\frac{4*3}{52*51}$.
We can also use conditional probability (which we will have to use in 
the next part):
\[P(K \text{ on 2nd and } K \text{ on 1st}) = P(\text{K on 2nd} | 
\text{K on 1st}) P(\text{K on 1st}) = \frac{3}{51}*\frac{4}{52}\].
 \end{solution}
 
\item The second card is a King without replacement?
\begin{solution}
\begin{equation}
\begin{split}
P(\text{K on 2nd}) &= P(K \text{ on 2nd }| K \text{ on 1st}) P(K \text{ on 1st}) \\
& + P(K \text{ on 2nd }| \text{ no $K$ on 1st}) P(\text{no $K$ on 1st}) \\
&= \frac{3}{51}*\frac{4}{52} + \frac{4}{51}*\frac{48}{52}  \\
&= \frac{4}{52}
\end{split}
\end{equation}

Note that this is the same as the $P(K\text{ on 1st})$, because a $K$ 
is equally likely to be anywhere in the deck.
 \end{solution}
 
\item \textbf{[EXTRA]} The nth card is a King without replacement ($n < 52$)?
\begin{solution} [1 cm]
The last problem is a hint that we can argue this by symmetry. Since 
we have no information about what any of the preceding cards were 
before the $n$th card, it is equally likely that the $n$th card is any of the 52 
possible cards, so the probability that it is a King is 
$\frac{4}{52}$.
\end{solution}

\end{enumerate}