\vspace{2mm}
\question Every tournament has a Hamiltonian path. 
(Recall that a Hamiltonian path is a path that visits each vertex 
exactly once.)

\begin{solution}[2.5 in]
\textit{Base Case}: For $n = 1$ nodes, there is a trivial 
Hamiltonian path. \newline
\textit{Inductive Hypothesis}: Assume that for a tournament 
with $n$ nodes, there is a Hamiltonian path.\newline
\textit{Inductive Step}: Consider a tournament $T$ with $n + 1$ nodes. 
Take an arbitrary node $x$, and remove it along with its incident 
edges. The resulting subgraph $T$\ensuremath{'} is also a tournament 
(each node in $T$\ensuremath{'} still shares some edge with every 
other node in $T$\ensuremath{'}). By the Inductive Hypothesis, 
there is some Hamiltonian path in $T$\ensuremath{'}. Let this 
Hamiltonian Path be $v_1, v_2, v_3, \dotsc , v_n$. Now we consider 
$T$. Note that since $T$ is a tournament, $x$ shares an edge with 
every other node in $T$. There are three possible cases:
\begin{enumerate}[label= ]
	\item \textit{Case 1}: Everybody beat $x$ (there is no edge 
	from $x$ to any node in $T$\ensuremath{'}). Then there is an 
	edge $(v_n, x)$. Thus, there is a Hamiltonian Path in $T$, 
	namely $v_1, v_2, v_3, \dotsc, v_n, x$.
	\item \textit{Case 2}: $x$ beat everybody (there is no edge 
	from any node in $T$\ensuremath{'} to $x$). Then there is an 
	edge $(x, v_1)$. Thus, there is a Hamiltonian Path in $T$, 
	namely $x, v_1, v_2, v_3, \dotsc , v_n$.
	 \item \textit{Case 3}: There is some $v_i$ that is the last 
	 person who beat $x$, in the ordering $v_1, \dotsc , v_n$. 
	 Note that $v_i$ must exist because we are not in Case 2, and 
	 $i \neq n$ because we are not in Case 1. Then since $v_i$ is 
	 the last person who beat $x$, there is an edge $(v_i, x)$, and 
	 an edge $(x, v_{i+1})$. Thus, there is a Hamiltonian path in $T$, 
	 namely $v_1, v_2, v_3, \dotsc , v_i, x, v_{i+1}, \dotsc , v_n$.
These are the only possible cases, so it must be that $T$ has a Hamiltonian Path.
\end{enumerate}
Therefore by induction, any tournament has a Hamiltonian Path. \newline
\end{solution}

