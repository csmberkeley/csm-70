\question Let $G=(V, E)$ be an undirected graph. Match the term with the definition. \newline
\noindent\fbox{\begin{minipage}{\dimexpr\textwidth-2\fboxsep-2\fboxrule\relax}
\vspace{3mm}
\hspace{.5 cm} Walk \hspace{3.5 cm} Cycle  \hspace{3.5 cm} Tour \hspace{3.5 cm} Path\hspace{.5 cm}
\vspace{.9mm}
\end{minipage}}

\fbox{\begin{minipage}{\dimexpr\textwidth-2\fboxsep-2\fboxrule\relax}
 \line(1,0){150} Walk that starts and ends at the same node \\
 \line(1,0){150} Sequence of edges. \\
 \line(1,0){150} Sequences of edges with possibly repeated vertex or edge. \\
 \line(1,0){150} Sequence of edges that starts and ends on the same vertex and 
does not repeat vertices (except the first and last) 
\end{minipage}}

\begin{solution}
 \center{\underline{tour}} Walk that starts and ends at the same node \\
 \center{\underline{path}} Sequence of edges. \\
 \center{\underline{walk}} Sequences of edges with possibly repeated vertex or edge.\\
 \center{\underline{cycle}} Sequence of edges that starts and ends on the same vertex and 
does not repeat vertices (except the first and last)
\end{solution}

\question  Suppose we want to represent a round-robin tennis tournament in which every player plays one match against every other player. How might we represent this using a graph?
\begin{solution}[1 in]
Have a node for each player. For every pair of players $(u, v)$, place an edge from the winner to the loser. This kind of graph is referred to as a tournament. 

This is similar to a complete graph, but it is directed.
\end{solution}

\question  What is a simple path?
\begin{solution}[1 in]
A sequence of edges where the vertices are necessarily distinct.
\end{solution}